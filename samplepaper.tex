% This is samplepaper.tex, a sample chapter demonstrating the
% LLNCS macro package for Springer Computer Science proceedings;
% Version 2.20 of 2017/10/04
%
\documentclass[runningheads]{llncs}
%
\usepackage{graphicx}
% Used for displaying a sample figure. If possible, figure files should
% be included in EPS format.
%
% If you use the hyperref package, please uncomment the following line
% to display URLs in blue roman font according to Springer's eBook style:
% \renewcommand\UrlFont{\color{blue}\rmfamily}
\usepackage{hyperref}

\begin{document}
%
\title{Music streaming and song recommendations using ML algorithms}
%
%\titlerunning{Abbreviated paper title}
% If the paper title is too long for the running head, you can set
% an abbreviated paper title here
%
\author{Anthony M. Schomer}
%
\authorrunning{A. Schomer.}
% First names are abbreviated in the running head.
% If there are more than two authors, 'et al.' is used.
%
\institute{Northwest Missouri State University, Maryville MO 64468, USA \\
\email{tony.schomer@gmail.com}}
%
\maketitle              % typeset the header of the contribution
%
\begin{abstract}
This capstone project investigates the algorithms used by music streaming services to recommend similar songs to enhance user experiences. The focus is on how platforms like Spotify and Apple Music use listeners preferences including Likes, dislikes, and other relevant information to create personalized playlists tailored to individual tastes. This project uses open datasets such as Spotify Million Playlist Dataset, Spotify Web API, and Musicbrain.org's extensive library of databases. Using smart computer programs to create a test system that suggests songs based on user behavior and preferences. It looks at how current song recommendations systems work. Machine learning methods. such as, collaborative filtering and content-based analysis to build test recommendation systems. This project also will address challenges within the current algorithms to help avoid common issues. One significant issue to avoid is users hearing the same song over and over again. These findings will help improve the music discovery online and suggest ways to innovate and enhance recommendations for listeners and introduce them to new artists. 

\keywords{music \and streaming \and recommendations \and data \and user experience}
\end{abstract}
%
%
%
\section{Introduction}

Listening to music is something many people do daily. Whether it is at the gym, at work, commuting from place to place. Most people use their phones to access streaming music services.

\subsection{Goals of this Research} 
How music streaming services use algorithms to recommend similar songs to users. 
Analyzing the processes of recommendations, the goal is to understand how these platforms personalize playlists and suggest new music that is similar to what the listeners' would enjoy.

\subsection{The following are the phases of implementation for the Project}
\begin{enumerate}
\item Research and Data Collection
    \begin{enumerate}
        \item Study existing music streaming platforms, Spotify, Apple Music).
        \item Gather open datasets (Spotify Million Playlist Datasets, Spotify Web API, Musicbrainz.org).
        \item Analyze user behavior data and song characteristics.
    \end{enumerate}
\item Algorithm Analysis
    \begin{enumerate}
        \item Examine current song recommendation systems and identify key factors influencing recommendations. 
        \item Study collaborative filtering and content-based analysis techniques. 
    \end{enumerate}
\item Development of Test System
    \begin{enumerate}
        \item Design and implement a prototype recommendation system using machine learning methods. 
        \item Integrate user behavior and preferences into the system. 
    \end{enumerate}
\item Addressing Possible Challenges
    \begin{enumerate}
        \item Develop strategies to enhance recommendation variety and avoid songs playing over and over. 
    \end{enumerate}
\item Analysis and Conclusion
    \begin{enumerate}
        \item Interpret findings from the test system and develop suggestions to improve music discovery.
        \item Discuss implications for listeners and new artists. 
    \end{enumerate}
\item Final Report and Presentation
    \begin{enumerate}
        \item Compile research and present findings and music streaming recommendations.
    \end{enumerate}
\end{enumerate}


\section{Data Collection and Analysis}

\subsection{Data Sources}
The data for this project was collected from several key sources:

\begin{itemize}
    \item \textbf{Spotify Million Playlist Dataset}: This dataset contains a large collection of playlists created by users on Spotify, which can be used to analyze user preferences and music trends.
    \item \textbf{Spotify Web API}: This API allows access to Spotify's music catalog, user playlists, and other relevant data, enabling real-time data retrieval for analysis.
    \item \textbf{MusicBrainz}: An open music encyclopedia that collects music metadata, which can be useful for enriching the dataset with artist and album information.
\end{itemize}

\subsection{Data Format and Structure}
The datasets utilized in this project are primarily in CSV format, which facilitates easy manipulation and analysis. The structure of the datasets includes various columns such as:

\begin{itemize}
    \item \texttt{track\_id}: Unique identifier for each track.
    \item \texttt{artist\_name}: Name of the artist.
    \item \texttt{album\_name}: Name of the album.
    \item \texttt{genre}: Genre of the music track.
    \item \texttt{tempo}: The speed of the track measured in beats per minute (BPM).
    \item \texttt{danceability}: A measure of how suitable a track is for dancing based on musical elements.
\end{itemize}

These attributes are crucial for analyzing user preferences and improving song recommendation algorithms.

\subsection{Data Extraction and Processing}
The process of extracting and processing data involved several steps:

1. **Data Extraction**:
   - Data was retrieved using APIs (like the Spotify Web API) to programmatically fetch playlists or track details based on specific queries.
   - Datasets were downloaded from repositories (such as Kaggle) in CSV format for offline analysis.

2. **Data Processing**:
   - **Data Cleaning**: 
     - Duplicate entries were removed based on unique identifiers such as \texttt{track\_id}.
     - Missing values were handled using imputation techniques; for instance, numerical features like tempo were filled with median values, while records with excessive missing categorical data (e.g., genre) were removed.
   - **Normalization**: Numerical features such as danceability were standardized to ensure consistency across analyses.
   - **Data Transformation**: Categorical variables were converted into numerical formats where necessary (e.g., one-hot encoding for genres).

3. **Tools Used**:
   - The primary tools utilized during this process included:
     - **Python with pandas**: For data manipulation and cleaning.
     - **NumPy**: For numerical operations and handling arrays.
     - **Jupyter Notebooks**: To document the data processing steps interactively.

An example code snippet illustrating some of these processes is shown below:

\begin{verbatim}
import pandas as pd

# Load the dataset
data = pd.read_csv('spotify_data.csv')

# Remove duplicates
data = data.drop_duplicates(subset='track_id')

# Fill missing values
data['tempo'].fillna(data['tempo'].median(), inplace=True)

# Normalize 'danceability' feature
data['danceability'] = (data['danceability'] - data['danceability'].min()) / (data['danceability'].max() - data['danceability'].min())

# One-hot encode 'genre'
data = pd.get_dummies(data, columns=['genre'])
\end{verbatim}

\subsection{Relevance to Research Question and Variable Identification}

Our data cleaning process directly supports our investigation into how music streaming services use algorithms for song recommendations. The cleaned dataset allows us to analyze factors influencing recommendations, including:

\begin{itemize}
    \item Track audio features (tempo, key, mode, danceability, etc.)
    \item User listening history and preferences
    \item Artist and genre information
\end{itemize}

We identify the following key variables:

\textbf{Independent Variables (Predictors):}
\begin{itemize}
    \item Track audio features (tempo, key, mode, danceability, energy, etc.)
    \item Artist popularity
    \item Genre
    \item Release date
    \item User's listening history
    \item User's saved tracks and playlists
\end{itemize}

\textbf{Dependent Variables (Outcomes):}
\begin{itemize}
    \item Likelihood of a track being recommended
    \item User engagement with recommended tracks (e.g., play count, skip rate)
    \item Diversity of recommendations
\end{itemize}

By analyzing the relationships between these variables, we aim to understand how current recommendation systems work and identify potential improvements. For example, we can investigate how audio features correlate with user preferences or how the inclusion of less popular artists in recommendations affects user engagement.

The extensive cleaning and preprocessing of our dataset ensures that we have reliable, consistent data to build and test our recommendation models. This process helps us avoid biases that could arise from incomplete or inconsistent data, thereby increasing the validity of our findings and the potential impact of our research on improving music streaming recommendation algorithms.

\subsection{Final Report and Presentation}
The final report will compile all research findings, methodologies employed during the project, and recommendations for enhancing music streaming services. It will also include visualizations representing user behavior patterns and algorithm performance metrics.

% ---- Bibliography ----
\bibliographystyle{splncs04}
\bibliography{mybibliography}

\section{Additional Resources}
The following resources were utilized and referenced throughout this project:

\begin{itemize}
    \item \href{https://www.kaggle.com/datasets/shubhendra/million-playlist-dataset}{Kaggle: Spotify Million Playlist Dataset}
    \item \href{https://developer.spotify.com/documentation/web-api/}{Spotify Web API Documentation}
    \item \href{https://musicbrainz.org/}{MusicBrainz Database}
    \item \href{https://www.acm.org/publications/policies/duplicate-publication}{ACM: Duplicate Publication Policy}
    \item \href{https://www.overleaf.com/project}{Overleaf Project}
    \item \textbf{GitHub Repository:} \href{https://github.com/anythonyschomer/Capstone-Project-Report}{Capstone Project Report}
\end{itemize}

% Any additional dashed information or notes can go here
\vspace{-1em} % Space before additional notes
\rule{\textwidth}{0.4pt} % Horizontal line for separation
\vspace{-0.5em} % Space before additional notes
This project was conducted by Anthony M. Schomer at Northwest Missouri State University.

\end{document}